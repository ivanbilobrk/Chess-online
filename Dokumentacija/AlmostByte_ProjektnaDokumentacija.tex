
\documentclass{article}

\usepackage[croatian]{babel}
\usepackage{amssymb}
\usepackage{amsmath}
\usepackage{txfonts}
\usepackage{mathdots}
\usepackage{titlesec}
\usepackage{array}
\usepackage{lastpage}
\usepackage{etoolbox}
\usepackage{tabularray}
\usepackage{color, colortbl}
\usepackage{adjustbox}
\usepackage{geometry}
\usepackage[classicReIm]{kpfonts}
\usepackage{hyperref}
\usepackage{fancyhdr}

\usepackage{float}
\usepackage{setspace}
\restylefloat{table}
\usepackage{graphicx}
\usepackage[table]{xcolor}
\graphicspath{ {./slike/} }


\linespread{1.3} %razmak između redaka

\geometry{a4paper, left=1in, top=1in,}  %oblik stranice

\hypersetup{ colorlinks, citecolor=black, filecolor=black, linkcolor=black,	urlcolor=black }   %izgled poveznice

%Promjena teksta za dugačke tablice
\DefTblrTemplate{contfoot-text}{normal}{Nastavljeno na idućoj stranici}
\SetTblrTemplate{contfoot-text}{normal}
\DefTblrTemplate{conthead-text}{normal}{(Nastavljeno)}
\SetTblrTemplate{conthead-text}{normal}
\DefTblrTemplate{middlehead,lasthead}{normal}{Nastavljeno od prethodne stranice}
\SetTblrTemplate{middlehead,lasthead}{normal}

%podesavanje zaglavlja i podnožja

\pagestyle{fancy}
\lhead{Programsko inženjerstvo}
\rhead{Šahisti}
\lfoot{AlmostByte}
\cfoot{\thepage}
\rfoot{\today}
\renewcommand{\headrulewidth}{0.2pt}
\renewcommand{\footrulewidth}{0.2pt}
\begin{document}

\newenvironment{packed_enum}{
	\begin{enumerate}
		\setlength{\itemsep}{0pt}
		\setlength{\parskip}{0pt}
		\setlength{\parsep}{0pt}
	}{\end{enumerate}}

\newenvironment{packed_item}{
	\begin{itemize}
		\setlength{\itemsep}{0pt}
		\setlength{\parskip}{0pt}
		\setlength{\parsep}{0pt}
	}{\end{itemize}}
	
		\begin{titlepage}
		\begin{center}
			\vspace*{\stretch{1.0}} %u kombinaciji s ostalim \vspace naredbama definira razmak između redaka teksta
			\LARGE Programsko inženjerstvo\\
			\large Ak. god. 2022./2023.\\
			
			\vspace*{\stretch{3.0}}
			
			\huge Šahisti\\
			\Large Dokumentacija, Rev. \textit{2}\\
			
			\vspace*{\stretch{12.0}}
			\normalsize
			Grupa: \textit{AlmostByte}\\
			Voditelj: \textit{Ivan Bilobrk}\\
			
			
			\vspace*{\stretch{1.0}}
			Datum predaje: \textit{13. 1. 2023.}\\
			
			\vspace*{\stretch{4.0}}
			
			Nastavnik: \textit{Igor Stančin}\\
			
		\end{center}
		
		
	\end{titlepage}

	\tableofcontents
	\eject
		
	\section{Dnevnik promjena dokumentacije}	
		\begin{longtblr}[
		label=none
		]{
			width = \textwidth, 
			colspec={|X[2]|X[13]|X[3]|X[3]|}, 
			rowhead = 1
		}
		\hline
		\textbf{Rev.}	& \textbf{Opis promjene/dodatka} & \textbf{Autori} & \textbf{Datum}\\[3pt] \hline
		0.1 & Preuzet predložak i izrađena gruba kopija predloška dokumentacije.	& Anteo \newline Vukasović & 18.10.2022. 		\\[3pt] \hline 
		0.2	& Napisan i dodan opis use caseova.& Anteo \newline Vukasović & 29.10.2022.	\\[3pt] \hline 
		0.3 & Napisani i dodani funkcionalni zahtjevi.  & Anteo \newline Vukasović & 03.11.2022. \\[3pt] \hline 
		0.4 & Dodani use case dijagrami. & Mijo Rajič & 05.11.2022. \\[3pt] \hline 
		0.5 & Napisan i dodan opis projektnog zadatka. & Anteo \newline Vukasović & 06.11.2022. \\[3pt] \hline 
		0.6 & Dodani sekvencijski dijagrami te njihovi opisi. & Altea Božić, Tina Jureško i Lara \newline Mahalec & 09.11.2022. \\[3pt] \hline 
		0.7 & Napisan i dodan opis baze podataka. &  Lara \newline Mahalec & 11.11.2022. \\[3pt] \hline
		0.8 & Napisan i dodan opis arhitekture sustava. & Anteo \newline Vukasović & 17.11.2022. \\[3pt] \hline
		0.9 & Napisani i dodani dijagrami razreda te odgovarajući opisi. & Altea Božić, Tina Jureško i Mijo Rajić & 18.11.2022. \\[3pt] \hline
		0.10 & Dodan popis sastanaka sa glavnim temama te ažurirana tablica aktivnosti prvog ciklusa & Anteo \newline Vukasović & 18.11.2022. \\[3pt] \hline  
		0.11 & Kompletirana tablica promjena dokumentacije za prvi ciklus. & Anteo \newline Vukasović & 18.11.2022. \\[3pt] \hline  
		\textbf{1.0} & \textbf{Verzija samo s bitnim dijelovima za 1. ciklus} & \textit{verificirao \newline cijeli tim} & 18.11.2022. \\[3pt] \hline
		1.1 & Dijagram stanja & Tina Jureško & 11.1.2023. \\[3pt] \hline 
		1.2 & Dijagram aktivnosti & Tina Jureško & 11.1.2023. \\[3pt] \hline
		1.3 & Dijagram komponenti & Altea Božić & 12.1.2023. \\[3pt] \hline
		1.4 & Dijagram razmještaja & Altea Božić & 12.1.2023. \\[3pt] \hline
		1.5 & Korištene tehnologije i alati & Anteo \newline Vukasović & 13.1.2023. \\[3pt] \hline 
		1.6 & Ispitivanje programskog rješenja & Lara \newline Mahalec & 13.1.2023. \\[3pt] \hline
		1.7 & Upute za puštanje u pogon & Ivan Bilobrk & 13.1.2023. \\[3pt] \hline
		1.8 & Zaključak i daljnji rad & Anteo \newline Vukasović & 13.1.2023. \\[3pt] \hline
		1.9 & Literatura i ažuriranje poglavlja aktivnosti grupe & Anteo \newline Vukasović & 13.1.2023. \\[3pt] \hline
		\textbf{2.0} & \textbf{Konačna verzija} & \textit{verificirao \newline cijeli tim} & 13.1.2023. \\[3pt] \hline
	\end{longtblr}

	\eject
	
	\section{Opis projektnog zadatka}
		Cilj ovog projektnog zadatka je izrada funkcionalne web aplikacije koju će koristiti članovi šahovskog kluba. Ova web aplikacija olakšala bi trenerima u šahovskom klubu organizaciju i vođenje treninga i natjecanja svojih učenika, a učenicima bi pružila platformu na kojoj bi mogli okušati svoja znanja i vještine putem dnevnih taktika. Na taj način svi korisnici mogu doraditi i unaprijediti svoja znanja i vještine u divnoj šahovskoj igri. \\ 
		Aplikacija je dizajnirana primarno za članove šahovskog kluba te će samo oni imati pristup punoj funkcionalnosti aplikacije. No to ne sprječava sve one zainteresirane da dobiju uvid u svijet šaha putem ove aplikacije. \\
		Prilikom pokretanja same aplikacije, korisnika se vodi na stranicu s 3 opcije:
		\begin{itemize}
			\item Registracija
			\item Prijava u sustav
			\item  Nastavi kao gost
		\end{itemize}
		Ukoliko korisnik želi nastaviti bez ikakve prijave ili registracije te odabere posljednju opciju, vodi ga se na početnu stranicu koja ne nudi sve opcije kao što bi u normalnom načinu rada, no ipak nudi dosta zabavnog sadržaja. Korisnik u ovom načinu rada može:
		\begin{itemize}
			\item Riješiti dnevnu taktiku 
			\item Pročitati novosti i zanimljivosti 
			\item Pregledati rang listu članova 
		\end{itemize}
		Novosti i zanimljivosti je rubrika koju treneri ureduju po svojoj volji. Koncipirana je kao svojevrsna lista članaka čiji sadržaj bi bile vijesti o radu i rezultatima kluba, vijesti iz svijeta šaha i šahovskih natjecanja diljem svijeta, crtice iz povijesti, analize legendarnih šahovskih partija i slično. 
		
		Dnevna taktika je kratka šahovska zagonetka, najčešće od 1 do 5 poteza, u kojoj je cilj naći optimalnu kombinaciju poteza u danom scenariju šahovske igre. Primjer jedne takve zagonetke možemo vidjeti na raznim šahovskim platformama kao što su lichess(link), chess.com(link) i slični. U našoj aplikaciji, dnevna taktika bilježi vrijeme rješavanja, te točnost ponuđenog rješenja. Rješavanjem dnevne taktike, neregistriranog korisnika se obavijesti o ispravnosti njegova rješenja, no njegovo rješenje se nigdje ne pohranjuje, niti taj korisnik sam može biti uvršten u rang listu. Za te funkcionalnosti korisnik mora biti član, a član može postati registracijom.\\
		
		\includegraphics[width=\columnwidth]{lichess_primjer}
		\begin{center}
			\textit{Slika 2.1: Primjer šahovske zagonetke}
		\end{center}
		
		Registracija nudi korisnicima koji žele postati članovi šahovskog kluba mogućnost da tu želju ostvare. Odabirom te opcije korisnika se vodi na formular u kojem korisnik unosi svoje vlastite podatke. Ukoliko su svi podatci ispravno uneseni, korisnik se pohranjuje u bazu podataka te postaje punopravni član šahovskog kluba. Čestitke!
		
		Prijava u sustav je slična registraciji, jedina razlika je ta što njoj pristupaju postojeći članovi kluba, stoga moraju unijeti ispravne podatke koji već postoje u bazi podataka šahovskog kluba kako bi im se omogućio pristup aplikaciji. 
		
		Ukoliko je korisnik prijavljen u sustav te se radi o članu šahovskog kluba koji nije trener, korisnik u aplikaciji može: 
		\begin{itemize}
			\item Pregledati osobne podatke 
			\item Riješiti dnevnu taktiku 
			\item Pročitati novosti i zanimljivosti 
			\item Pregledati rang listu članova 
			\item Prijaviti se na treninge 
			\item Prijaviti se na turnire
			\item Biti dio rang liste svih članova 
			\item Prijaviti rješenja dnevne taktike 
			\item Platiti članarinu i pregledati svoja prijašnja plaćanja 
		\end{itemize}
		
		Treća i četvrta točka prethodne liste se ne razlikuju prema funkcionalnosti od funkcionalnosti koju za iste komponente imaju neregistrirani članovi. 
		
		Rješavanje dnevne taktike je isto kao i kod neregistriranih korisnika, no ono što se odvija nakon samog rješavanja jest drugačije. Kod registriranih članova, rješenja se pohranjuju u bazu podataka. Na temelju toga kreira se rang lista svih članova kluba. Rang lista uzima u obzir sve riješene taktike te 3 parametra tih rješenja; točnost, brzinu i težinu same taktike. 
		
		Osim rješavanja dnevne taktike članovi imaju opciju prijavljivanja rješenja. Ukoliko smatraju kako rješenje nije ispravno, korisnici odabiru opciju za prijavu dnevne taktike te obavljaju proces podnošenja prijave koji se sastoji od 3 koraka: 
		\begin{packed_enum}
			\item Unos novih poteza za koje član smatra da su ispravni 
			\item Tekstualni opis tih rješenja, iznošenje svojih misli iza priloženih novih rješenja 
			\item Odabir trenera kojem se šalje zahtjev 
		\end{packed_enum}
		Ukoliko trener prihvati novo rješenje, ono se uzima kao ispravno te se rang liste ažuriraju prema ovim rješenjima.  
		
		Član se unutar naše aplikacije može prijavljivati na treninge i turnire objavljene na aplikaciju od strane trenera. Odabirom opcije za prijavu na turnir/trening, član odlazi na stranicu gdje vidi popis svih dostupnih treninga i turnira. Ukoliko nema nikakvih konflikata ili prekršenih ograničenja o maksimalnom broju dopuštenih članova u treningu/turniru, člana se prijavljuje na dani trening/turnir. 
		
		Odabirom opcije za pregled osobnih podataka, član na jednom mjestu vidi svoje osnovne osobne podatke koje je unio u trenutku registracije te svu svoju aktivnost na aplikaciji. 
		
		Svaki član mora platiti članarinu kako bi mogao i dalje biti punopravni član, a naša aplikacija nudi opciju plaćanja. Odabirom opcije za plaćanje članarine, člana se vodi na popis svih svojih računa. Ukoliko na tom popisu ima neplaćenih, člana se o tome obavijesti te odabirom pojedinog neplaćenog računa korisnik može platiti taj račun unošenjem osnovnih podataka potrebnih za plaćanje. Za člana je bitno da uredno plaća svoje račune jer je moguće da mu administrator(više o njemu kasnije) ograniči pristup aplikaciji sve dok ne plati sve neplaćene članarine.\\ 
		
		
		Osim članova, u šahovskom klubu postoje i treneri. Oni su zaduženi za objavljivanje sadržaja i aktivnosti na aplikaciju koje članovi kasnije mogu koristiti. Treneri na aplikaciji mogu:
		
		\begin{itemize}
			\item Pregledati osobne podatke 
			\item Objavljivati i ažurirati dnevne taktike 
			\item Objavljivati novosti i zanimljivosti 
			\item Objavljivati treninge 
			\item Organizirati i objaviti turnire 
			\item Revidirati zaprimljene prijave o neispravnim rješenjima dnevnih taktika 
		\end{itemize} 
		
		Prva točka je identična kao kod članova, iste funkcionalnosti imaju i jedni i drugi. 
		
		Treneri mogu objavljivati novosti i zanimljivosti na aplikaciju te postavljati nove dnevne taktike. Kod svake od navedenih rubrika u aplikaciji, za trenere će biti ponuđena opcija dodavanja novog sadržaja. Primjerice, kod unosa novosti, treneru će se otvoriti prostor za unošenje teksta u kojem trener upisuje što želi. Nakon što je gotov, trener potvrdi svoju namjeru objavljivanja odabirom prikladnog gumba te se taj tekst dodaje u aplikaciju. 
		
		Treninge i turnire trener dodaje odabirom opcije za dodavanje novog treninga/turnira. Odabire termin početka i vrijeme trajanja te potencijalno ograničenje maksimalnog broja sudionika. Ukoliko su svi kriteriji zadovoljeni, aktivnost se dodaje u bazu podataka i prikazuje se unutar aplikacije. Ukoliko termin nije ispravan te dolazi do kolizije s nekom drugom aktivnosti, trenera se vraća na ponovni odabir ispravnog termina. 
		
		U prijašnjem dijelu teksta smo vidjeli kako članovi mogu prijavljivati dnevne taktike za koje misle da imaju bolja rješenja od onih koja su ponuđena. Odabirom na opciju pristigle prijave, trener može pregledati imaju li kakvih prijava od članova. Ukoliko imaju, mogu pogledati cijeli tekst prijave i proučiti ponovno dnevnu taktiku. Smatraju li kako je prijava neutemeljena treneri je samo odbacuju. Na ako je prijava valjana, trener može promijeniti rješenje dnevne taktike te ju na taj način ažurirati. Tim činom se ujedno i ažuriraju rang liste.\\ 
		
		
		Osim do sada navedenih korisnika aplikacije, postoji i još jedna vrsta korisnika a to je administrator. Njegove ovlasti su slične kao kod trenera, no još dodatno proširene. On može vidjeti sav sadržaj aplikacije, ali može ga i ukloniti i mijenjati po volji, bez obzira je li on autor tog sadržaja. Uz to, administrator ima opciju zabrane pristupa korisnicima. Ova akcija se može dogoditi iz proizvoljnih razloga no istaknimo jedan poseban scenarij; član uredno ne plaća svoju članarinu. Administrator može vidjeti pregled svih plaćanja svakog člana te ako utvrdi da pojedini član ima previše neplaćenih računa može mu ograničiti pristup aplikaciji. Taj član će od funkcionalnosti jedino moći vidjeti svoja plaćanja i platiti svoje neplaćene račune, i tako sve dok ne plati račune. U tom slučaju, administrator može maknuti restrikcije tom članu.
		\eject 
		
	\section{Specifikacija programske potpore}
		\subsection{Funkcionalni zahtjevi}
		\textit{Dionici ovog sustava su predsjednik šahovskog kluba, administrator sustava, razvojni tim te članovi kluba}\\
		
		\noindent \textit{Aktori ovog sustava su neregistrirani korisnici, registrirani članovi, registrirani treneri te administratori kao inicijatori, te baza podataka kao sudionik. }\\
		
		\noindent \textbf{Dionici:}
		
		\begin{packed_enum}
			
			\item Predsjednik kluba (naručitelj) 
			\item Članovi kluba 
			\begin{packed_enum}
				\item Treneri
				\item Polaznici
			\end{packed_enum}
			\item Administrator 
			\item Razvojni tim 
			
		\end{packed_enum}
	
		\noindent \textbf{Aktori:}
		\begin{packed_enum}
			\item Neregistrirani korisnik(inicijator) može:
			\begin{packed_enum}
				\item Obaviti registraciju i postati članom kluba 
				\item Pregledati novosti postavljene od strane trenera/administratora na aplikaciju 
				\item Rješavati dnevnu taktiku 
				\item Vidjeti rang listu uspješnosti rješavanja dnevnih taktika svih članova
			\end{packed_enum}
			\item Registrirani član(inicijator) može:
			\begin{packed_enum}
				\item Pregledati svoje osobne podatke 
				\item Obaviti plaćanje i pregled računa(članarine) 
				\item Prijaviti se na treninge postavljene na stranicu od strane trenera 
				\item Prijaviti se na turnire postavljene na stranicu od strane trenera
				\item Rješavati dnevne taktike uz pohranjivanje rješenja u bazu podataka te ažuriranje rang liste svih članova na temelju točnosti i brzine rješavanja taktike 
				\item Prijaviti greške u dnevnoj taktici na način da unese i opiše nove poteze za koje smatra da su pravo rješenje te pošalje svoja zapažanja proizvoljnom treneru na reviziju 
			\end{packed_enum}
			\item Trener(inicijator) može:
			\begin{packed_enum}
				\item Pregledati svoje osobne podatke 
				\item Postavljati nove dnevne taktike na aplikaciju 
				\item Postavljati novosti na aplikaciju 
				\item Postavljati raspored svojih treninga na aplikaciju te omogućiti članovima prijavu na iste 
				\item Organizirat turnire za članove te omogućiti članovima prijavu na iste 
				\item Revidirati zaprimljene prijave pogreški u dnevnim taktikama te ukoliko je potrebno ažurirati ih 
			\end{packed_enum}
			\item Administrator(inicijator) može:
			\begin{packed_enum}
				\item Imati uvid u sve podatke na aplikaciji te opciju uređivanja i dodavanja novih sadržaja
				\item Zabraniti pristup aplikaciji određenim članovima/trenerima 
				\item Pregledati plaćanja svih članova te ukoliko ima neplaćenih računa određenim članovima zabraniti pristup dok ne obavi plaćanja
			\end{packed_enum}
			\item Baza podataka(sudionik):
			\begin{packed_enum}
				\item Pohranjuje sve osobne podatke o registriranim članovima
				\item Pohranjuje sve akcije registriranih članova
				\item Pohranjuje status o potencijalnoj zabrani pristupa nekom od članova
			\end{packed_enum}
		\end{packed_enum}
		\eject
		
		\subsubsection{Obrasci uporabe}
		\noindent \textbf{Opis obrazaca uporabe:}\\
		
		\noindent {\textbf{UC1 - Registracija}}
		\begin{packed_item}
			
			\item \textbf{Glavni sudionik: }Neregistrirani korisnik
			\item  \textbf{Cilj:} Stvaranje korisničkog računa za neregistriranog korisnika
			\item  \textbf{Sudionici:} Baza podataka
			\item  \textbf{Preduvjet:} -
			\item  \textbf{Opis osnovnog tijeka:}
			
			\item[] \begin{packed_enum}
				\item Odabir gumba za registraciju
				\item Ispunjavanje ponuđenog formulara
				\item Pohranjivanje podataka u bazu podataka
				\item Otvaranje početne stranice 
			\end{packed_enum}
			
			\item  \textbf{Opis mogućih odstupanja:}
			
			\item[] \begin{packed_item}
				
				\item[2.a] Odabir već zauzetog korisničkog imena i/ili e-maila, unos korisničkog podatka u nedozvoljenom formatu ili pružanje neispravnoga e-maila
				\item[] \begin{packed_enum}
					\item Sustav obavještava korisnika o greški i vraća ga na registraciju 
					\item Korisnik ispravlja grešku i uspješno obavlja registraciju ili ne uspijeva dok ne odustane
				\end{packed_enum}
			\end{packed_item}
		\end{packed_item}
	
		\noindent {\textbf{UC2 - Prijava u sustav}}
		\begin{packed_item}
			
			\item \textbf{Glavni sudionik: }Registrirani korisnik
			\item  \textbf{Cilj:} Verifikacija postojanja korisničkog računa i mogućnost pristupa većem broju sadržaja na stranici
			\item  \textbf{Sudionici:} Baza podataka
			\item  \textbf{Preduvjet:} -
			\item  \textbf{Opis osnovnog tijeka:}
			
			\item[] \begin{packed_enum}
				\item Odabir gumba za prijavu u sustav 
				\item Popunjavanje danog formulara 
				\item Provjera postojanja danih podataka u samoj bazi podataka 
				\item Otvaranje početne stranice 
			\end{packed_enum}
			
			\item  \textbf{Opis mogućih odstupanja:}
			
			\item[] \begin{packed_item}
				
				\item[2.a] Korisniku je onemogućen pristup aplikaciji
				\item[2.b] Unos neispravnog korisničkog imena ili lozinke 
				\item[] \begin{packed_enum}
					\item Sustav obavještava korisnika o greški i vraća ga na prijavu u sustav  
					\item Korisnik ispravlja grešku i uspješno obavlja registraciju ili ne uspijeva dok ne odustane
				\end{packed_enum}
			\end{packed_item}
		\end{packed_item}	
		
		\noindent {\textbf{UC3 - Pregled novosti}}
		\begin{packed_item}
			
			\item \textbf{Glavni sudionik: }Korisnik
			\item  \textbf{Cilj:} Mogućnost pregledavanja novosti postavljenih na stranicu od strane ovlaštenih korisnika(trener/administrator)
			\item  \textbf{Sudionici:} Baza podataka
			\item  \textbf{Preduvjet:} -
			\item  \textbf{Opis osnovnog tijeka:}
			
			\item[] \begin{packed_enum}
				\item Korisnik otvaranjem glavne stranice vidi sve novosti postavljene na stranici  
				\item Klikom na novost dobiva potpuni pregled same novosti  
			\end{packed_enum}
		\end{packed_item}	
		
				\noindent {\textbf{UC4 - Rješavanje dnevne taktike}}
		\begin{packed_item}
			
			\item \textbf{Glavni sudionik: }Korisnik
			\item  \textbf{Cilj:}  Otvaranje postavljene dnevne taktike i prijavljivanje rješenja
			\item  \textbf{Sudionici:} Baza podataka
			\item  \textbf{Preduvjet:} -
			\item  \textbf{Opis osnovnog tijeka:}
			
			\item[] \begin{packed_enum}
				\item Otvaranje dnevne taktike i prikaz prosječne ocjene taktike 
				\item Prijava rješenja i njegova pohrana u bazu podataka te ažuriranje rang listi(pohrana i ažuriranje samo ako je korisnik registriran) 
				\item Nakon predaje rješenja korisnik dobije obavijest o točnosti rješenja te točno rješenje ukoliko ga sam nije ponudio  
			\end{packed_enum}
		\end{packed_item}	
		
		\noindent {\textbf{UC5 - Pregled rang-liste}}
		\begin{packed_item}
			
			\item \textbf{Glavni sudionik: }Korisnik
			\item  \textbf{Cilj:}  Pregled težinske liste temeljene na točnosti i brzini rješavanja dnevnih taktika. Rang-lista sastavljena je od svih članova. 
			\item  \textbf{Sudionici:} Baza podataka
			\item  \textbf{Preduvjet:} -
			\item  \textbf{Opis osnovnog tijeka:}
			
			\item[] \begin{packed_enum}
				\item Korisnik pritiskom na gumb odlazi na stranicu rang-liste 
				\item Otvara mu se popis svih članova sortiranih na temelju točnosti i brzini rješavanja dnevnih taktika 
			\end{packed_enum}
		\end{packed_item}
		
		\noindent {\textbf{UC6 - Pregled osobnih podatka}}
		\begin{packed_item}
			
			\item \textbf{Glavni sudionik: }Registrirani korisnik
			\item  \textbf{Cilj:} Pregled osobnih podatka
			\item  \textbf{Sudionici:} Baza podataka
			\item  \textbf{Preduvjet:} Korisnik je prijavljen u sustav
			\item  \textbf{Opis osnovnog tijeka:}
			
			\item[] \begin{packed_enum}
				\item Korisnik odabire gumb za prikaz osobnih podataka i aktivnosti na stranici 
				\item Podatci se prikazuju u aplikaciji 
			\end{packed_enum}
		\end{packed_item}		
		
		\noindent {\textbf{UC7 - Plaćanje članarine}}
		\begin{packed_item}
			
			\item \textbf{Glavni sudionik: }Član
			\item  \textbf{Cilj:} Član plaća svoj mjesečni iznos članarine 
			\item  \textbf{Sudionici:} Baza podataka
			\item  \textbf{Preduvjet:} Korisnik je prijavljen u sustav\\
			\item  \textbf{Opis osnovnog tijeka:}
			
			\item[] \begin{packed_enum}
				\item Korisnik odabire gumb za plaćanje članarine  
				\item Na otvorenoj stranici se prikazuju neplaćeni računi(ukoliko ih ima) 
				\item Korisnik odabere račun te potvrdi plaćanje 
			\end{packed_enum}
		\end{packed_item}		
		
		\noindent {\textbf{UC8 - Prijava na trening}}
		\begin{packed_item}
			
			\item \textbf{Glavni sudionik: }Član
			\item  \textbf{Cilj:} Prijavljivanje korisnika na odabrani trening 
			\item  \textbf{Sudionici:} Baza podataka
			\item  \textbf{Preduvjet:} Korisnik je prijavljen u sustav
			\item  \textbf{Opis osnovnog tijeka:}
			
			\item[] \begin{packed_enum}
				\item Korisnik odabire gumb za prijave na treninge  
				\item Aplikacija prikazuje popis dostupnih treninga iz baze podataka  
				\item Korisnik odabire trening i njegov odabir se pohranjuje u bazu podataka 
			\end{packed_enum}
		\end{packed_item}
	
		\noindent {\textbf{UC9 - Prijava na turnir}}
		\begin{packed_item}
			
			\item \textbf{Glavni sudionik: }Član
			\item  \textbf{Cilj:} Prijavljivanje korisnika na odabrani turnir 
			\item  \textbf{Sudionici:} Baza podataka
			\item  \textbf{Preduvjet:} Korisnik je prijavljen u sustav
			\item  \textbf{Opis osnovnog tijeka:}
			
			\item[] \begin{packed_enum}
				\item Korisnik odabire gumb za prijave na turnire  
				\item Aplikacija prikazuje popis dostupnih turnira iz baze podataka  
				\item Korisnik odabire turnir i njegov odabir se pohranjuje u bazu podataka 
			\end{packed_enum}
		\end{packed_item}

		\noindent {\textbf{UC10 - Dodjeljivanje ocjene dnevnoj taktici}}
		\begin{packed_item}
			
			\item \textbf{Glavni sudionik: }Član
			\item  \textbf{Cilj:} Dodjeljivanje ocjene dnevnoj taktici od strane člana 
			\item  \textbf{Sudionici:} Baza podataka
			\item  \textbf{Preduvjet:} Korisnik je prijavljen u sustav te je riješio dnevnu taktiku 
			\item  \textbf{Opis osnovnog tijeka:}
			
			\item[] \begin{packed_enum}
				\item Nakon rješavanja dnevne taktike korisniku se otvara opcija ocjenjivanja taktike   
				\item Korisnik unosi ocjenu koja se zatim pohranjuje u bazu podataka  
			\end{packed_enum}
		\end{packed_item}
	
		\noindent {\textbf{UC11 - Prijava pogreške}}
		\begin{packed_item}
			
			\item \textbf{Glavni sudionik: }Član
			\item  \textbf{Cilj:} Član prijavljuje pogrešku u taktici ukoliko smatra da postoji optimalnije rješenje  
			\item  \textbf{Sudionici:} Baza podataka
			\item  \textbf{Preduvjet:} Korisnik je prijavljen u sustav te je riješio dnevnu taktiku 
			\item  \textbf{Opis osnovnog tijeka:}
			
			\item[] \begin{packed_enum}
				\item Korisniku se nakon rješavanja taktike nudi opcija prijavljivanja pogreške   
				\item Korisnik odabire gumb te se otvara forma za prijavljivanje pogreške 
				\item Nakon ispunjavanja forme, korisnik podnosi zahtjev za promjenom taktike koji se može odobriti ili odbiti  
			\end{packed_enum}
		\end{packed_item}
		
		\noindent {\textbf{UC12 - Unos novih poteza }}
		\begin{packed_item}
			
			\item \textbf{Glavni sudionik: }Član
			\item  \textbf{Cilj:} Korisnik predlaže nove poteze kao rješenje   
			\item  \textbf{Sudionici:} Baza podataka
			\item  \textbf{Preduvjet:} Korisnik je prijavljen u sustav te je riješio dnevnu taktiku 
			\item  \textbf{Opis osnovnog tijeka:}
			
			\item[] \begin{packed_enum}
				\item Nakon odabira prijavljivanja greške korisnik unosi svoje nove poteze kao rješenja  
			\end{packed_enum}
		\end{packed_item}
	
		\noindent {\textbf{UC13 - Opis novih poteza }}
		\begin{packed_item}
			
			\item \textbf{Glavni sudionik: }Član
			\item  \textbf{Cilj:} Korisnik daje obrazloženje predloženog rješenja taktike   
			\item  \textbf{Sudionici:} Baza podataka
			\item  \textbf{Preduvjet:} Korisnik je prijavljen u sustav te je riješio dnevnu taktiku 
			\item  \textbf{Opis osnovnog tijeka:}
			
			\item[] \begin{packed_enum}
				\item Nakon unesenih novih poteza korisnik daje pismeno objašnjenje tih poteza  
			\end{packed_enum}
		\end{packed_item}
	
		\noindent {\textbf{UC14 - Odabir trenera za revidiranje }}
		\begin{packed_item}
			
			\item \textbf{Glavni sudionik: }Član
			\item  \textbf{Cilj:} Korisnik odabire trenera koji će pregledati njegov prijedlog o promjeni taktike    
			\item  \textbf{Sudionici:} Baza podataka
			\item  \textbf{Preduvjet:} Korisnik je prijavljen u sustav te je riješio dnevnu taktiku 
			\item  \textbf{Opis osnovnog tijeka:}
			
			\item[] \begin{packed_enum}
				\item Nakon predanog opisa poteza korisniku se otvara opcija odabira medu trenerima u bazi 
				\item Korisnik odabire trenera kojem zatim dolazi taj isti zahtjev za revizijom  
			\end{packed_enum}
		\end{packed_item}
		
		\noindent {\textbf{UC15 - Postavljanje dnevne taktike  }}
		\begin{packed_item}
			
			\item \textbf{Glavni sudionik: }Trener
			\item  \textbf{Cilj:} Trener postavlja dnevnu taktiku koju ostali korisnici zatim mogu rješavati    
			\item  \textbf{Sudionici:} Baza podataka
			\item  \textbf{Preduvjet:} Korisnik je prijavljeni trener 
			\item  \textbf{Opis osnovnog tijeka:}
			
			\item[] \begin{packed_enum}
				\item Trener kreira dnevnu taktiku s njenim zadanim rješenjima  
				\item Postavljanje taktike na stranicu te njeno pohranjivanje u bazu podataka 
			\end{packed_enum}
		\end{packed_item}
		
		\noindent {\textbf{\\UC16 - Slaganje rasporeda vlastitih treninga   }}
		\begin{packed_item}
			
			\item \textbf{Glavni sudionik: }Trener
			\item  \textbf{Cilj:} Trener odabire termine treninga i postavlja ih na aplikaciju   
			\item  \textbf{Sudionici:} Baza podataka
			\item  \textbf{Preduvjet:} Korisnik je prijavljeni trener 
			\item  \textbf{Opis osnovnog tijeka:}
			
			\item[] \begin{packed_enum}
				\item Trener odabire opciju postavljanja novog treninga  
				\item Trener postavlja vrijeme početka i trajanja samog treninga, te njegov kratki opis 
				\item Nakon što je termin potvrđen kao slobodan, trening se postavlja na aplikaciju i pohranjuje u bazu podataka 
			\end{packed_enum}
			
			\item  \textbf{Opis mogućih odstupanja:}
			
			\item[] \begin{packed_item}
				
				\item[2.a] Termin treninga je u preklapanju s drugim aktivnostima
				\item[] \begin{packed_enum}
					\item Trenera se vraća na ponovno postavljanje termina dok ne unese dostupni termin ili odustane od postavljanja treninga 
				\end{packed_enum}
			\end{packed_item}
		\end{packed_item}
	
		\noindent {\textbf{UC17 - Organiziranje turnira  }}
		\begin{packed_item}
			
			\item \textbf{Glavni sudionik: }Trener
			\item  \textbf{Cilj:} Trener odabire osnovne parametre turnira i postavlja ga na aplikaciju   
			\item  \textbf{Sudionici:} Baza podataka
			\item  \textbf{Preduvjet:} Korisnik je prijavljeni trener 
			\item  \textbf{Opis osnovnog tijeka:}
			
			\item[] \begin{packed_enum}
				\item Trener odabire opciju postavljanja novog termina  
				\item Trener postavlja vrijeme početka i trajanja turnira, njegov opis i maksimalni broj sudionika 
				\item Nakon što je termin potvrđen kao slobodan, turnir se postavlja na aplikaciju i pohranjuje u bazu podataka 
			\end{packed_enum}
		\end{packed_item}
		
		\noindent {\textbf{UC18 - Objavljivanje novosti }}
		\begin{packed_item}
			
			\item \textbf{Glavni sudionik: }Trener
			\item  \textbf{Cilj:} Trener postavlja novosti/zanimljivosti/crtice iz povijesti na aplikaciju drugim korisnicima na čitanje   
			\item  \textbf{Sudionici:} Baza podataka
			\item  \textbf{Preduvjet:} Korisnik je prijavljeni trener 
			\item  \textbf{Opis osnovnog tijeka:}
			
			\item[] \begin{packed_enum}
				\item Trener odabire opciju dodavanja novosti na aplikaciji  
				\item Trener unosi novost te potvrđuje svoje postavljanje  
				\item Novost se objavljuje na aplikaciji te pohranjuje u bazu podataka  
			\end{packed_enum}
		\end{packed_item}
		
		\noindent {\textbf{UC19 - Revidiranje pogreški  }}
		\begin{packed_item}
			
			\item \textbf{Glavni sudionik: }Trener
			\item  \textbf{Cilj:} Trener nakon zaprimanja zahtjeva za revizijom dnevne taktike, pregledava taktiku i donosi odluku o potencijalnoj promjeni   
			\item  \textbf{Sudionici:} Baza podataka
			\item  \textbf{Preduvjet:} Korisnik je prijavljeni trener te ima pristigli zahtjev za revizijom taktika
			\item  \textbf{Opis osnovnog tijeka:}
			
			\item[] \begin{packed_enum}
				\item Trener odabire gumb za pristigle zahtjeve te prelazi na novu stranicu  
				\item Trener odabire jedan od pristiglih zahtjeva te mu se otvara cijeli sadržaj zahtjeva   
				\item Nakon revizije trener odbacuje zahtjev ili ga prihvaća  
			\end{packed_enum}
		\end{packed_item}
	
		\noindent {\textbf{UC20 - Mijenjanje dnevne taktike  }}
		\begin{packed_item}
			
			\item \textbf{Glavni sudionik: }Trener
			\item  \textbf{Cilj:} Trener zaključuje da je zahtjev ispravan te mijenja dnevnu taktiku  
			\item  \textbf{Sudionici:} Baza podataka
			\item  \textbf{Preduvjet:} Korisnik je prijavljeni trener te je obavio reviziju taktike
			\item  \textbf{Opis osnovnog tijeka:}
			
			\item[] \begin{packed_enum}
				\item Nakon potvrde da se radi o valjanom zahtjevu, trener mijenja rješenja taktike  
				\item Ažurirana taktika se objavljuje na aplikaciju i pohranjuje u  bazu podatka  
				\item Ažuriraju se rang-liste na temelju novih rješenja   
			\end{packed_enum}
		\end{packed_item}
		
		\noindent {\textbf{UC21 - Dodavanje svih sadržaja  }}
		\begin{packed_item}
			
			\item \textbf{Glavni sudionik: }Administrator
			\item  \textbf{Cilj:} Administrator može dodavati sve sadržaje na aplikaciju  
			\item  \textbf{Sudionici:} Baza podataka
			\item  \textbf{Preduvjet:} Korisnik je prijavljeni administrator
			\item  \textbf{Opis osnovnog tijeka:}
			
			\item[] \begin{packed_enum}
				\item Administrator ima opciju dodavati bilo koji od prije navedenih sadržaja   
				\item Administratoru su kod svih prije navedenih sadržaja dostupne opcije za dodavanje koje on može koristiti kao i prije navedeni korisnici 
			\end{packed_enum}
		\end{packed_item}
	
		\noindent {\textbf{UC22 - Brisanje svih sadržaja  }}
		\begin{packed_item}
			
			\item \textbf{Glavni sudionik: }Administrator
			\item  \textbf{Cilj:} Administrator može ukloniti bilo kakav sadržaj sa stranice  
			\item  \textbf{Sudionici:} Baza podataka
			\item  \textbf{Preduvjet:} Korisnik je prijavljeni administrator
			\item  \textbf{Opis osnovnog tijeka:}
			
			\item[] \begin{packed_enum}
				\item Administrator kod bilo kojeg sadržaja vidi opciju za uklanjanje istog    
				\item Ukoliko odabere opciju uklanjanja, sadržaj se miče sa prikaza aplikacije, no ostaje pohranjen u bazi podataka  
			\end{packed_enum}
		\end{packed_item}
		
		\noindent {\textbf{UC23 - Zabrana pristupa}}
		\begin{packed_item}
			
			\item \textbf{Glavni sudionik: }Administrator
			\item  \textbf{Cilj:} Administrator može zabraniti pristup bilo kojem registriranom korisniku 
			\item  \textbf{Sudionici:} Baza podataka
			\item  \textbf{Preduvjet:} Korisnik je prijavljeni administrator
			\item  \textbf{Opis osnovnog tijeka:}
			
			\item[] \begin{packed_enum}
				\item Administrator odabire opciju za zabranu pristupa
				\item Uz prikazani popis zabranjenih korisnika, administrator može dodati novog korisnika pritiskom na gumb za dodavanje novih osoba u popis 
				\item Pritiskom na gumb otvori se formular u kojeg administrator unosi osobne podatke korisnika kojem želi zabraniti pristup 
				\item Ukoliko uneseni korisnik zapravo postoji, njegovo ime se dodaje na popis zabranjenih korisnika u bazi podataka te mu se zabranjuje pristup stranici  
			\end{packed_enum}
			
			\item  \textbf{Opis mogućih odstupanja:}
			
			\item[] \begin{packed_item}
				
				\item[3.a] Administrator je unio korisničke podatke koji ne postoje u bazi svih korisnika 
				\item[] \begin{packed_enum}
					\item Administratora se vraća na ponovni upis podataka u formular sve dok ne unese postojeće podatke ili dok ne odustane 
				\end{packed_enum}
			\end{packed_item}
		\end{packed_item}
	
		\noindent {\textbf{UC24 -  Pregled svih plaćanja  }}
		\begin{packed_item}
			
			\item \textbf{Glavni sudionik: }Administrator
			\item  \textbf{Cilj:} Administrator pregledava sva plaćanja članarina svih članova 
			\item  \textbf{Sudionici:} Baza podataka
			\item  \textbf{Preduvjet:} Korisnik je prijavljeni administrator
			\item  \textbf{Opis osnovnog tijeka:}
			
			\item[] \begin{packed_enum}
				\item Korisnik za svakog člana ima opciju pregleda računa     
				\item Odabirom te opcije otvara se popis svih računa te napomena uz svaki račun je li on plaćen ili ne 
			\end{packed_enum}
		\end{packed_item}
	
		\noindent {\textbf{UC25 - Zabrana svih funkcionalnosti na temelju neplaćenih računa  }}
		\begin{packed_item}
			
			\item \textbf{Glavni sudionik: }Administrator
			\item  \textbf{Cilj:} Administrator zabranjuje funkcionalnosti članovima koji nisu platili članarinu dok ona nije plaćena
			\item  \textbf{Sudionici:} Baza podataka, prijavljeni korisnik
			\item  \textbf{Preduvjet:} Korisnik je prijavljeni administrator
			\item  \textbf{Opis osnovnog tijeka:}
			
			\item[] \begin{packed_enum}
				\item Korisnik otvara pregled plaćanja člana     
				\item Ukoliko član nema sve plaćene račune omogućen je gumb za zabranu pristupa člana na temelju neplaćenih računa 
				\item Ukoliko je gumb pritisnut, tom članu se dopušta prijava u sustav, no sve funkcionalnosti osim plaćanja računa su mu uskraćene 
				\item Nakon plaćanja računa, funkcionalnosti se vraćaju 
			\end{packed_enum}
		\end{packed_item}
		
		\eject
		
		\textbf{Dijagrami obrazaca uporabe}\\\\
		\includegraphics[width=\columnwidth]{korisnik_uc_dijagram}
		\begin{center}
			\textit{Slika 3.1: Dijagram obrasca uporabe, korisnička funkcionalnost}
		\end{center}
		\eject
		
		\includegraphics[width=\columnwidth]{clanovi_i_treneri_uc_dijagram}
		\begin{center}
			\textit{Slika 3.2: Dijagram obrasca uporabe, funkcionalnost člana i trenera}
		\end{center}
		\eject
		
		\includegraphics[width=\columnwidth]{administrator_uc_dijagram}
		\begin{center}
			\textit{Slika 3.3: Dijagram obrasca uporabe, funkcionalnost administratora}
		\end{center}
		\eject
		
		\subsubsection{Sekvencijski dijagrami}
		\textbf{Obrazac uporabe UC4 - Rješavanje dnevne taktike}\\
		Kad član pokrene rješavanje dnevne taktike, poslužitelj započinje mjeriti vrijeme, a nakon toga u bazu podataka prosljeđuje izmjereno vrijeme odnosno ishod rješavanja dnevne taktike.
		Aplikacija uz to kreira težinsku rang listu članova na temelju povijesnog učinka u rješavanju taktika, izmjerenog vremena za pojedini član te težine dnevne taktike.
		Ako član zamijeti pogrešku u dnevnoj taktici to prijavljuje unoseći nove poteze i opis novih poteza, zatim svaka prijavljena pogreška odlazi na revidiranje proizvoljno odabranom treneru. Ako se trener složi s novim rješenjem, ono se mora promijeniti na navedenoj taktici i automatski se moraju revidirati rang liste pri čemu se dodjeljuju bodovi onim članovima koji su ponudili točno rješenje, a uklanjaju onima koji su ponudili staro rješenje.
		\eject
		\includegraphics[width=\columnwidth]{rjesavanje_dnevnih_taktika}
		\begin{center}
			\textit{Slika 3.4: Sekvencijski dijagram za UC4}
		\end{center}
		\eject
		\textbf{Obrazac uporabe UC17 - Organiziranje turnira}\\
		Trener šalje zahtjev za pregled slobodnih termina kako bi mogao odabrati termin kada će se održati turnir. Poslužitelj dohvaća slobodne termine i prikazuje ih. Nakon što trener odabere termin, provjerava se ispravnost te se sprema odabir u bazu podataka nakon čega se prikazuje poruka koja ukazuje na uspješan odabir i rezervaciju termina. Trener može opcionalno promijeniti određene pojedinosti turnira. Ukoliko se na to odluči, trener šalje zahtjev za izmjenu pojedinosti. Poslužitelj dohvaća dosad određene pojedinosti treneru na izmjenu. Nakon što trener izmjeni željene pojedinosti, one se spremaju u bazu podataka. Na kraju se prikazuje poruka koja ukazuje na uspješnu izmjenu pojedinosti.
		\eject
		\includegraphics[width=\columnwidth]{organizacija_turnira}
		\begin{center}
			\textit{Slika 3.5: Sekvencijski dijagram za UC17}
		\end{center}
		\eject
		\textbf{Obrazac uporabe UC24 - Pregled svih plaćanja}\\
		Administrator šalje zahtjev poslužitelju za prikaz transakcija pojedinog člana. Poslužitelj pristupa bazi podataka, dohvaća tražene transakcije i prikazuje ih. Ako transakcije ne postoje ili nisu provedene u cijelosti, administrator može članu zabraniti pristup svim funkcionalnostima aplikacije osim uplate članarine.
		\eject
		\includegraphics[width=\columnwidth]{prikaz_transakcija}
		\begin{center}
			\textit{Slika 3.6: Sekvencijski dijagram za UC24}
		\end{center}
		\eject
		\subsection{Ostali zahtjevi}
		\begin{packed_enum}
			\item Sustav mora omogućiti paralelno korištenje aplikacije više korisnika u isto vrijeme bez ikakvih međusobnih interferencija
			\item Aplikacija mora biti maksimalno responzivna, korisnik ne smije puno čekati na odgovor ili prikaz sadržaja od strane sustava
			\item Sustav treba biti intuitivan i jednostavan za korištenje
			\item Neispravno korištenje ne smije rezultirati narušavanjem funkcionalnosti
			\item Korisnik treba biti spriječen pristupu određenih dijelova aplikacije ukoliko nije ovlašten
			\item Valute unutar sustava prikazane su u Eurima
			\item Nadogradnje sustava ne smiju narušavati postojeće funkcionalnosti sustava
			\item Pristup aplikaciji mora biti omogućen iz javne mreže
			\item Ukoliko je korisniku zabranjen pristup odlukom administratora, ne smije imati nikakve mogućnosti korištenja i treba biti obaviješten o tome, a ukoliko je zabranjen pristup zbog neplaćenih računa, treba mu biti sve ograničeno osim plaćanja računa
			\eject
		\end{packed_enum}
		
		\section{Arhitektura i dizajn sustava}
		\subsection{Opis arhitekture sustava}
		Arhitektura sustava može se podijeliti na 3 dijela:
		\begin{packed_enum}
			\item Prikaz web aplikacije korisniku (front-end)
			\item Web poslužitelj (back-end)
			\item Baza podataka
		\end{packed_enum}
	
		Našem proizvodu pristupa se putem web preglednika. Putem web preglednika, naša web aplikacija napisana u svojim programskim jezicima se prevodi u oblik koji je korisniku jasniji i prirodniji. To korisniku omogućuje korištenje svih funkcionalnosti naše aplikacije.\\
		Web poslužitelj temelj je rada naše aplikacije. On prima zahtjeve poslane od strane korisnika te vraća informacije koje je korisnik zatražio. Također, poslužitelj ima pristup bazi podataka u kojoj su pohranjeni svi podatci relevantni za rad naše aplikacije. Više o bazi podataka možete pročitati u poglavlju 4.2 (Baza podataka). Komunikacija između korisnika i poslužitelja odvija se pomoću HTTP protokola.\\
		Web aplikacija je korisniku prikazana na način da mu se omogući što lakši rad i interakcija s mogućnostima same aplikacije. Odabirom jedne od ponuđenih opcija na stranici, web preglednik šalje zahtjeve za potrebnim podatcima i/ili resursima s web poslužitelja/baze podataka na što mu poslužitelj odgovara. Ukoliko nije došlo do nikakvih nepredviđenih poteškoća, korisniku se prikazuje zatraženi sadržaj.\\
		Front-end naše aplikacije razvijen je u JavaScriptu ili točnije Reactu, open-source JavaScript biblioteci koja se koristi za jednostavnije kreiranje korisničkih sučelja. Kao dodatnu pomoć u nekim dijelovima korišten je i CSS.\\
		Back-end naše aplikacije razvijen je u JavaScriptu.\\
		Baza podataka razvijena je u pgAdmin 4, GUI-u razvijen za upravljanje i korištenje PostgreSQL sustavom upravljanja bazi podataka. \\
		Razvojno okruženje korišteno za razvoj back-end i front-end dijela aplikacije je Visual Studio Code.\\
		Arhitektura sustava modelirana je prema MVC konceptu. Taj koncept se sastoji od 3 dijela; modela koji kontrolira radom i pravilima aplikacije, prima podatke od controllera te upravlja podatcima, viewa koji predstavlja bilo kakav grafički ili drugi prikaz podataka te controllera, komponente koja prima zahtjeve korisnika i prosljeđuje ih drugim dijelovima sustava.\\
		Osim do sada navedenih značajki, naš sustav koristi JWT tehnologiju prenošenja podataka u obliku JSON objekata. JWT je skraćeno od JSON Web Token, a u našoj aplikaciji se koriste 2 vrste tokena, access token i refresh token. Access token služi za pristup resursima, a vremenski je ograničen. Refresh token ima dulje vrijeme trajanja od access tokena te služi za pridruživanje novog access tokena korisniku nakon isteka prijašnjeg access tokena. Na taj način korisnik se ne mora ponovno prijavljivati u sustav, no nakon eventualnog isteka refresh tokena, korisnik će ipak morati obaviti ponovnu prijavu u sustav. 
		
		\eject
		\subsection{Baza podataka}
		Za potrebe našeg sustava koristit ćemo relacijsku bazu podataka koja svojom strukturom olakšava modeliranje stvarnog svijeta. Gradivna jedinka baze je relacija, odnosno tablica koja je definirana svojim imenom i skupom atributa. Zadaća baze podataka je brza i jednostavna pohrana, izmjena i dohvat podataka za daljnju obradu.Baza podataka ove aplikacije sastoji se od sljedećih entiteta:
		
			\begin{packed_item}
			
			\item {Users}
			\item  {Training} 
			\item  {Tournament}
			\item  {Scheduled training}
			\item  {Scheduled tournament}
			\item  {News}
			\item  {Daily tactics}
			\item  {Score}
			\item  {Reported mistake}
			\item  {Membership}
		\end{packed_item} 
		
	\large \textbf{Opis tablica}\\
	
	\textbf{Users} Ovaj entitet sadržava sve važne informacije o korisniku aplikacije. Sadrži atribute: userID, role, name, surname, username, email i pwdHash. Ovaj entitet u vezi je \textit{One-to-Many} s entitetom Training, Tournament, scheduledTraining, scheduledTournament, News, dailyTactics, Membership, reportedMistake i Score preko jedinstvenog identifikatora korisnika.
		
		
		
\setlength{\arrayrulewidth}{0.5mm}
\setlength{\tabcolsep}{10pt}
\renewcommand{\arraystretch}{1.5}		
		


	
	\begin{center}
    \begin{tabular}{ | l | l | l | p{5cm} |}
    \hline
    \multicolumn{3}{|c|}{Users}  \\ \hline
   \cellcolor{green!25}userID & INT & jedinstveni identifikator korisnika\\ \hline
    role & VARCHAR & uloga korisnika (trainer, member, admin)\\ \hline
    same & VARCHAR & ime korisnika \\ \hline
    surname & VARCHAR & prezime korisnika\\\hline
   email & VARCHAR & e-mail adresa korisnika\\ \hline
   pwdHash & INT & hash lozinke\\ \hline
    \end{tabular}
\end{center}
\eject

\textbf{Training} Ovaj entitet sadržava sve važne informacije o šahovskom treningu pojedinog trenera. Sadrži atribute: trainingID, trainerID, trainingStartTimeDate i trainingDurationMin. Ovaj entitet u vezi je \textit{Many-to-One} sa entitetom Users preko jedinstvenog identifikatora korisnika i \textit{One-to-Many} sa entitetom scheduledTraining.
		\\






  \begin{center}
    \begin{tabular}{ | l | l | l | p{5cm} |}
    \hline
    \multicolumn{3}{|c|}{Training}  \\ \hline
   \cellcolor{green!25}trainingID & INT & jedinstveni identifikator treninga \\ \hline  
   \cellcolor{blue!15}trainerID & INT & jedinstveni identifikator trenera \\ \hline
      trainingStartTimeDate  & TIMESTAMP & vrijeme početka treninga \\ \hline
    trainingDurationMin & INT & trajanje treninga \\ \hline
    \end{tabular}
\end{center}    
\bigskip
\bigskip
\bigskip


\textbf{Tournament} Ovaj entitet sadržava sve važne informacije o šahovskom turniru. Sadrži atribute: tournamentID, trainerID, tournamentStartTimeDate, tournamentDurationMin i participantsNo. Ovaj entitet u vezi je \textit{Many-to-One} sa entitetom Users preko jedinstvenog identifikatora korisnika i \textit{One-to-Many} sa entitetom scheduledTournament.
		\\
		
		
		

	\begin{center}
      \begin{tabular}{ | l | l | l | p{5cm} |}
    \hline
    \multicolumn{3}{|c|}{Tournament}  \\ \hline
    \cellcolor{green!25}tournamentID  & INT & jedinstveni identifikator turnira\\ \hline
    \cellcolor{blue!15}trainerID & INT & identifikator trenera \\ \hline
      tournamentStartTimeDate  &  TIMESTAMP  & vrijeme početka turnira \\ \hline
    tournamentDurationMin & INT & trajanje turnira\\\hline
    participantsNo   & INT & broj natjecatelja\\ \hline
    \end{tabular}
\end{center}
	\eject
	
	
	
	
	
	\textbf{scheduledTournament} Pomoću ovog entiteta zapisujemo koji član je prijavljen na koji turnir. Sadrži atribute: memberID i tournamentID. Ovaj entitet u vezi je \textit{Many-to-One} sa entitetom Tournament preko jedinstvenog identifikatora turnira i \textit{Many-to-One} sa entitetom Users preko jedinstvenog identifikatora korisnika.
		\\
		\bigskip
	
	
\begin{center}
    \begin{tabular}{ | l | l | l | p{5cm} |}
    \hline
    \multicolumn{3}{|c|}{scheduledTournament}  \\ \hline
   \cellcolor{green!25}memberID & INT & jedinstveni identifikator člana \\ \hline
    \cellcolor{green!25}tournamentID & INT & jedinstveni identifikator turnira \\ \hline
    \end{tabular}
\end{center}
\bigskip
\bigskip


\textbf{scheduledTraining} Pomoću ovog entiteta zapisujemo koji član je prijavljen na koje treninge. Sadrži atribute: memberID i trainingID. Ovaj entitet u vezi je \textit{Many-to-One} sa entitetom Training preko jedinstvenog identifikatora treninga i \textit{Many-to-One} sa entitetom Users preko jedinstvenog identifikatora korisnika.
		\\



\begin{center}
    \begin{tabular}{ | l | l | l | p{5cm} |}
    \hline
    \multicolumn{3}{|c|}{scheduledTraining}  \\ \hline
   \cellcolor{green!25}memberID & INT & jedinstveni identifikator člana \\ \hline
    \cellcolor{green!25}tournamentID & INT & jedinstveni identifikator trninga \\ \hline 
    \end{tabular}
\end{center}
\bigskip
\bigskip


\textbf{News} Ovaj entitet sadržava sve važne informacije o objavljenoj novosti. Sadrži atribute: newsID, trainerID i content. Ovaj entitet u vezi je \textit{Many-to-One} sa entitetom Users preko jedinstvenog identifikatora korisnika.
		\\



\begin{center}
    \begin{tabular}{ | l | l | l | p{5cm} |}
    \hline
    \multicolumn{3}{|c|}{News}  \\ \hline
   \cellcolor{green!25}newsID & INT & jedinstveni identifikator novosti \\ \hline
    \cellcolor{blue!15}trainerID & INT & jedinstveni identifikator trenera \\ \hline
    content & VARCHAR & sadržaj novosti \\ \hline 
    \end{tabular}
\end{center}





\textbf{dailyTactics} Ovaj entitet sadržava sve važne informacije o dnevnoj taktici. Sadrži atribute: tacticID, trainerID, content i solution. Ovaj entitet u vezi je \textit{Many-to-One} sa entitetom Users preko jedinstvenog identifikatora korisnika, \textit{One-to-Many} sa entitetima reportedMistake i Score.
		\\



\begin{center}
    \begin{tabular}{ | l | l | l | p{5cm} |}
    \hline
    \multicolumn{3}{|c|}{dailyTactics}  \\ \hline
   \cellcolor{green!25}tacticID & INT & jedinstveni identifikator dnevne taktike \\ \hline
    \cellcolor{blue!15}trainerID & INT & jedinstveni identifikator trenera \\ \hline
   content & VARCHAR & opis dnevnog zadatka \\ \hline 
    solution & VARCHAR & opis rješenja \\ \hline 
    \end{tabular}
\end{center}
\bigskip
\bigskip



\textbf{Score} Pomoću ovog entiteta zapisujemo uspjeh rješavanja dnevne taktike. Sadrži atribute: memberID, tacticID, solvingTime i accuracy. Ovaj entitet u vezi je \textit{Many-to-One} sa entitetom Users preko jedinstvenog identifikatora korisnika i \textit{Many-to-One} sa entitetom dailyTactics preko jedinstvenog identifikatora dnevne taktike.
		\\



\begin{center}
    \begin{tabular}{ | l | l | l | p{5cm} |}
    \hline
    \multicolumn{3}{|c|}{Score}  \\ \hline
   \cellcolor{green!25}memberID & INT & jedinstveni identifikator člana \\ \hline
    \cellcolor{blue!15}tacticID & INT & jedinstveni identifikator dnevne taktike \\ \hline
    solvingTime & TIME & vrijeme rješavanja\\ \hline 
    accuracy & FLOAT & točnost rješenja\\ \hline
    \end{tabular}
\end{center}
\eject



\textbf{reportedMistake} Pomoću ovog entiteta zapisujemo prijavljenu pogrešku dnevne taktike. Sadrži atribute: memberID, tacticID, trainerID, preposedMove, moveDescription i isFixed. Ovaj entitet u vezi je \textit{Many-to-One} sa entitetom Users preko jedinstvenog identifikatora korisnika i \textit{Many-to-One} sa entitetom dailyTactics preko jedinstvenog identifikatora dnevne taktike.
		\\
		



\begin{center}
    \begin{tabular}{ | l | l | l | p{5cm} |}
    \hline
    \multicolumn{3}{|c|}{reportedMistake}  \\ \hline
   \cellcolor{green!25}memberID & INT & jedinstveni identifikator člana \\ \hline
    \cellcolor{green!25}tacticID & INT & jedinstveni identifikator dnevne taktike \\ \hline
     \cellcolor{blue!15}trainerID & INT & jedinstveni identifikator trenera \\ \hline
    preposedMove & VARCHAR & predloženo rješenje \\ \hline 
     moveDescription & VARCHAR & opis predloženog rješenja \\ \hline
     isFixed & BOOLEAN & oznaka je li greška rješena\\ \hline
    \end{tabular}
\end{center}
\bigskip
\bigskip



\textbf{Membership} Pomoću ovog entiteta zapisujemo plaćanje članarine članova. Sadrži atribute: memberID, periodStart, periodEnd i isPaid. Ovaj entitet u vezi je \textit{Many-to-One} sa entitetom Users preko jedinstvenog identifikatora korisnika.
		\\



\begin{center}
    \begin{tabular}{ | l | l | l | p{5cm} |}
    \hline
    \multicolumn{3}{|c|}{Membership}  \\ \hline
   \cellcolor{green!25}memberID & INT & jedinstveni identifikator člana \\ \hline
    periodStart & INT & početak peroda za koji se plaća članarina \\ \hline
    periodEnd & INT & kraj peroda za koji se plaća članarina \\ \hline
      isPaid & BOOLEAN & oznaka je li članarina plaćena\\ \hline
      
    \end{tabular}
\end{center}



	\includegraphics[width=\columnwidth]{ERbaze.png}
		\begin{center}
			\textit{Slika 4.1: E-R dijagram baze podataka}
		\end{center}

\eject

	\subsection{Dijagrami razreda}
	Back-end arhitektura našeg sustava prikazana je u sljedećim dijagramima označenim brojevima 4.2, 4.3 i 4.4.\\
	Prikazani dijagrami razreda sa slike 4.2 preslikavaju strukturu baze podataka u aplikaciji. Implementirane metode direktno komuniciraju s bazom podataka te vraćaju tražene podatke poput novosti, rang liste ili dnevnih taktika. \\
	

	\includegraphics[width=\columnwidth]{slike/dijagramRazreda.PNG}
		\begin{center}
			\textit{Slika 4.2: Dijagram razreda}
		\end{center}
	
	Razredi \textit{Member, Admin i Trainer} specifikacija su razreda \textit{User} pa s tom činjenicom nasljeđuju njegove public atribute i metode.
	
	Razred \textit{Member} predstavlja neregistriranog ili registriranog korisnika koji može rješavati dnevne taktike, a ako je atribut registered postavljen na true, može se prijaviti na turnire, treninge, ocjeniti dnevne taktike, prijaviti pogreške na istima ako ih ima te plaćati članarinu.
	
	Razred \textit{Admin} predstavlja administratora koji jedini ima dozvolu zabraniti pristup ostalim ulogama.
	\\
	
	Razred \textit{Trainer} predstavlja trenera koji sastavlja dnevne taktike, provodi provjeru prijavljenih pogrešaka dnevne taktike, sastavlja rasporede za treninge i organizira turnire te sastavlja novosti.
	
	Razred \textit{Training} i razred \textit{ScheduledTraining} u vezi su "cjelina-dio" pri čemu je razred Training agregat. Isti slučaj vrijedi i za razrede \textit{Tournament} i \textit{ScheduledTournament}.
	
	Razred\textit{ Membership} predstavlja skup atributa potrebnih za uvid o placenim članarinama, a razred \textit{Mistake} ima potrebne atribute i metode kojima se upravlja postupkom prijave pogreške, provjere i prihvaćanja ili odbijanja novog poteza.
	\\
	\\
	\\
	\\
	\\
	Na slici 4.3 prikazani su razredi koji reprezentiraju controller dio našeg sustava. 
	\includegraphics[width=\columnwidth]{controllers}
		\begin{center}
			\textit{Slika 4.3: Dijagram razreda - Controllers dio}
		\end{center}
	\eject
	Na slici 4.4 prikazan je dijagram razreda koji reprezentiraju Data trasnfer objects.
	\includegraphics[width=\columnwidth, height=18cm]{dto_class_diagram}
	\begin{center}
		\textit{Slika 4.4: Dijagram razreda - Data transfer objects}
	\end{center}
	DTO služi za razmjenu raznih zahtjeva i podataka između različitih dijelova sustava.
	
	\eject
	\subsection{Dijagram stanja}
	Dijagram stanja prikazuje stanja objekta te prijelaze iz jednog stanja u drugo temeljene na događajima. \\
	Na slici 4.5 prikazan je dijagram stanja za registriranog korisnika. \\
	\includegraphics[width=\columnwidth]{stanja}
	\begin{center}
		\textit{Slika 4.5: Dijagram stanja}
	\end{center}
	Nakon prijave, klijentu se prikazuje početna stranica na kojoj može pregledati novosti, rang listu te odigrati dnevnu taktiku.\\
	Klikom na "Osobni podatci" prikazuju mu se njegovi podatci koje može uređivati po želji.\\
	Registrirani korisnik ima i opcije prijave na treninge klikom na "Uplati\\ trening",kao i prijavu na turnir klikom na "Pregledaj nadolazeće turnire".
	Korisnik uplatu izvršava klikom na "Plati članarinu". \\
	U slučaju da je registrirani korisnik administrator, ima opciju pregleda svih članova kluba uz pomoć gumba "Klikni za pregled svih članova".
	\eject
	\subsection{Dijagram aktivnosti}
	Dijagram aktivnosti primjenjuje se za opis modela toka upravljanja ili toka podataka. Ne upotrebljava se za modeliranje događajima poticanog ponašanja.\\ 
	U modeliranju toka upravljanja svaki novi korak poduzima se nakon završenog prethodnog, a naglasak je na jednostavnosti. Na dijagramu aktivnosti sa slike 4.6 prikazan je proces uređivanja profila.\\
	\\
	\includegraphics[width=\columnwidth]{aktivnost}
	\begin{center}
		\textit{Slika 4.6: Dijagram aktivnosti}
	\end{center}
	Korisnik nakon uspješne prijave u sustav sa svojim korisničkim podatcima klikom na gumb "Uredi profil" započinje uređivanje nakon čega se pojavljuje prozorčić za upis novih podataka te klikom na "Submit" web-aplikacija šalje upit za spremanjem podataka bazi.\\
	Ako je uređivanje uspješno izvršeno, korisniku se prikazuju njegovi novi osobni podatci.
	\eject
	\subsection{Dijagram komponenti}
	Dijagram komponenti
	Slika 4.7 prikazuje dijagram komponenti kojim je vizualizirana organizacija te međuovisnost između implementacijskih komponenata te odnos programske potpore prema okolini.\\
	\\
	\includegraphics[width=\columnwidth]{komponente}
	\begin{center}
		\textit{Slika 4.7: Dijagram komponenti}
	\end{center}
	Postoje dva različita sučelja. Sučelje za dohvat HTML, CSS te Javascript datoteka služi za posluživanje datoteka s frontend dijela aplikacije, a koje su prikazane kao komponente koje su potrebne komponenti Router za njezino djelovanje (ovisnost). Router je komponenta koja određuje koja će datoteka biti poslužena na sučelju. Sučelje za JSON podatke omogućuje pristup REST API-u koji je zadužen za backend podatke aplikacije. React-view komponenta preko sučelja komunicira s web aplikacijom Šahisti te djeluje ovisno o korisnikovim akcijama.
	\eject
	
	\section{Implementacija i korisničko sučelje}
	\subsection{Korištene tehnologije i alati}
	U radu na našem projektu za međusobnu komunikaciju korištene su 3 aplikacije; Discord\footnote{\url{https://discord.com/}}, MS Teams\footnote{\url{https://www.microsoft.com/en-us/microsoft-teams/group-chat-software}} i WhatsApp\footnote{\url{https://www.whatsapp.com/}}.\\
	Upravljanje izvornim kodom ostvareno je sustavom Git\footnote{\url{https://git-scm.com/}}, a udaljeni repozitorij našeg projekta dostupan nam je bio na web platformi GitLab\footnote{\url{https://gitlab.com/}}.\\
	Kao razvojno okruženje korišten je Visual Studio Code\footnote{\url{https://code.visualstudio.com/}}, uređivač izvornog koda razvijen od strane Microsofta s Frameworkom, dostupan za Windowse, Linux te macOS. Visual Studio Code podržava rad u brojnim jezicima te nudi mogućnost preuzimanja mnoštva ekstenzija za omogućavanje i olakšavanje rada programerima stoga je nama pružio dobru platformu za rad.\\
	Backend dio stranice napisan je u jeziku JavaScript\footnote{\url{https://www.javascript.com/}}, a frontend dio stranice napisan je u Reactu\footnote{\url{https://reactjs.org/}}, biblioteci u JavaScriptu koja se koristi za izradu korisničkih sučelja.\\
	Baza podataka izrađena je pomoću PostgreSQL\footnote{\url{https://www.postgresql.org/}} sustava za upravljanja bazom podataka, a za lakšu vizualizaciju podatka korišten je i pgAdmin\footnote{\url{https://www.pgadmin.org/}}.  
	\eject
	\subsection{Ispitivanje programskog rješenja}
	\subsubsection{Ispitivanje komponenti}
	Test 1: Ispitivanje registracije\\
	Provedeno je ispitivanje uspješnosti registracije stvaranjem novog korisnika sa podatcima koji ne postoje u bazi podataka.  Provodi se test koji nakon klika gumba za registraciju provjera pokazuje li se obavijest “Uspješna registracija“. Ukoliko se korisniku prikazuje obavijest, registracija je uspješno izvršena.\\
	Prikaz koda ispitivanja:\\
	
	\includegraphics[width=\columnwidth]{registration}
	\begin{center}
		\textit{Slika 5.1}
	\end{center}
	Prikaz rezultata ispitivanja:\\
	
	\includegraphics[width=\columnwidth]{registration-true}
	\begin{center}
		\textit{Slika 5.2}
	\end{center}
	\eject
	Test 2: Ispitivanje prijave
	Provedeno je ispitivanje uspješnosti prijave u sustav upisivanjem podataka korisnika koji već postoje u sustavu. Provodi se test koji nakon klika gumba za prijavu provjerava trenutni URL. Ako se korisniku prikazuje profil, prijava je uspješno izvršena.
	
	Prikaz koda ispitivanja:\\
	
	\includegraphics[width=\columnwidth]{login_}
	\begin{center}
		\textit{Slika 5.3}
	\end{center}
	Prikaz rezultata ispitivanja:\\
	
	\includegraphics[width=\columnwidth]{login-true}
	\begin{center}
		\textit{Slika 5.4}
	\end{center}
	\eject
	Test 3: Ispitivanje pogreške prilikom prijave\\
	Ispituje se hoće li se prikazati obavijest o pogrešnom unosu prilikom unosa nepostojećih podataka. Nakon pogrešnog unosa provjerava se prikazuje li se obavijest sa tekstom “Pogreška User ne postoji“. Ukoliko je obavijest pronađena, test je ispravan.\\
	
	Prikaz koda ispitivanja:\\
	
	\includegraphics[width=\columnwidth]{wrongLogin}
	\begin{center}
		\textit{Slika 5.5}
	\end{center}
	Prikaz rezultata ispitivanja:\\
	
	\includegraphics[width=\columnwidth]{wrongLogin-true}
	\begin{center}
		\textit{Slika 5.6}
	\end{center}
	\eject
	Test 4: Ispitivanje dohvata točnih podataka o korisniku\\
	Ispituje se prikaz podataka korisniku nakon prijave, ako svi pročitani podatci odgovaraju očekivanim vrijednostima, test je ispravan.\\
	
	Prikaz koda ispitivanja:\\
	
	\includegraphics[width=\columnwidth]{userData}
	\begin{center}
		\textit{Slika 5.7}
	\end{center}
	Prikaz rezultata ispitivanja:\\
	
	\includegraphics[width=\columnwidth]{userData-true}
	\begin{center}
		\textit{Slika 5.8}
	\end{center}
	\eject
	Test 5: Ispitivanje dohvata točnih podataka o plaćanju\\
	Ispituje se prikaz podataka o plaćanju korisnika, nakon prijave prelazimo na stranicu za plaćanje i provjeravamo ima li upisanih transakcija. S obzirom da testiramo novog korisnika, očekivani rezultat je “Nema uplata“. Ukoliko pročitano odgovara očekivanom rezultatatu, test je ispravan.\\
	
	Prikaz koda ispitivanja:\\
	
	\includegraphics[width=\columnwidth]{paymentData}
	\begin{center}
		\textit{Slika 5.9}
	\end{center}
	Prikaz rezultata ispitivanja:\\
	
	\includegraphics[width=\columnwidth]{paymentData-true}
	\begin{center}
		\textit{Slika 5.10}
	\end{center}
	\eject
	Test 6: Ispitivanje pogreške prikikom rješavanja dnevne taktike\\
	Ispituje se hoće li se prikazati obavijes o pogrešnom rješenju dnevne taktike. Očekivani rezultat je “ Krivo rješenje. Pokušajte ponovo!“. Ukoliko izlaz odgovara oekivanom rješenju, test je ispravan.\\
	
		Prikaz koda ispitivanja:\\
	\includegraphics[width=\columnwidth]{dnevnaTaktika_}
	\begin{center}
		\textit{Slika 5.11}
	\end{center}
	Prikaz rezultata ispitivanja:\\
	\includegraphics[width=\columnwidth]{dnevnaTaktika-true}
	\begin{center}
		\textit{Slika 5.12}
	\end{center}
	\eject
	\subsubsection{Ispitivanje sustava}
	Test 1: ispitivanje registracije\\
	Preduvjeti: User name i email ne postoje u bazi\\
	Podatci: \\
		1.	Name: Prvi \\
		2.	Surname: Test\\
		3.	Username: PrviiTest\\
		4.	Email Adress: prviitest@gmail.com\\
		5.	Password:****\\
		6.	Confirm Password: ****\\
	Koraci:\\
		1. Kliknuti na link za registraciju\\
		2. Unijeti točne podatke u predviđena polja\\
		3. Kliknuti na gumb “Registriraj se!“\\
		4. Kliknuti na link za prijavu\\
	
	Očekivani rezultat: Registracija je uspješna, nakon unosa podataka korisnik se može prijaviti na svoj račun \\
	Stvarni izlaz: Jednak očekivanom rezultatu\\
	
	Prikaz rezultata za registraciju u Selenium IDE-u\\
	
	\includegraphics[width=\columnwidth]{registracija}
	\begin{center}
		\textit{Slika 5.13}
	\end{center}
	\eject
	Prikaz stranice prije i nakon registracije:\\
	
	\includegraphics[width=\columnwidth]{registracija-prije}
	\begin{center}
		\textit{Slika 5.14}
	\end{center}
	\includegraphics[width=\columnwidth]{registracija-nakon}
	\begin{center}
		\textit{Slika 5.15}
	\end{center}
	\eject
	Test 2: Ispitivanje prijave\\
	Preduvjeti: Korisnik je registriran i zapisan u bazi podataka\\
	Podatci: \\
	1.Username: PrviiTest\\
	2.Password: ****\\
	Koraci:\\
	1. Kliknuti na link za prijavu\\
	2. Unijeti točne podatke u predviđena polja\\
	3. Kliknuti na gumb “Prijavi se!“\\
	
	Očekivani rezultat: Prijava je uspješna, nakon unosa podataka korisnik može vidjeti svoje podatke\\
	Stvarni izlaz: Jednak očekivanom rezultatu\\
	Prikaz rezultata za registraciju u Selenium IDE-u:\\
	
	\includegraphics[width=\columnwidth]{login}
	\begin{center}
		\textit{Slika 5.16}
	\end{center}
	\eject
	Prikaz stranice prije i nakon prijave:\\
	
	\includegraphics[width=\columnwidth]{prijava-prije}
	\begin{center}
		\textit{Slika 5.17}
	\end{center}
	\includegraphics[width=\columnwidth]{prijava-nakon}
	\begin{center}
		\textit{Slika 5.18}
	\end{center}
	\eject
	
	Test 3: ispitivanje registracije s pogrešnim unosom\\
	Preduvjeti: User name ili email već postoje u bazi\\
	Podatci: \\
	7.	Name: Prvi \\
	8.	Surname: Test\\
	9.	Username: PrviiTest\\
	10.	Email Adress: prviitest@gmail.com\\
	11.	Password:****\\
	12.	Confirm Password: ****\\
	Koraci:\\
	1. Kliknuti na link za registraciju\\
	2. Unijeti točne podatke u predviđena polja\\
	3. Kliknuti na gumb “Registriraj se!“\\
	4. Izmjeniti već postojeće podatke\\
	
	Očekivani rezultat: Registracija nije uspješna, nakon unosa podataka javlja se greška “User name je već iskorišten!“ \\
	Stvarni izlaz: Jednak očekivanom rezultatu
	Prikaz rezultata za registraciju u Selenium IDE-u\\
	
	\includegraphics[width=\columnwidth]{registracija-greska}
	\begin{center}
		\textit{Slika 5.19}
	\end{center}
	\eject
	Prikaz stranice:\\
	
	\includegraphics[width=\columnwidth]{reg-greska}
	\begin{center}
		\textit{Slika 5.20}
	\end{center}
	\eject
	
	Test 4: ispitivanje stvaranja nove novosti\\
	Preduvjeti: Korisnik je admin ili trener\\
	Koraci:\\
	1. Kliknuti na ikonu za dodavanje novosti\\
	2. Unijeti točne podatke u predviđena polja\\
	3. Kliknuti na gumb “Spremi“\\
	
	Očekivani rezultat: Nova novost je unešena u bazu podataka\\  
	Stvarni izlaz: Jednak očekivanom rezultatu\\
	Prikaz rezultata za registraciju u Selenium IDE-u:\\
	
		\includegraphics[width=\columnwidth]{nova-novost}
	\begin{center}
		\textit{Slika 5.21}
	\end{center}
	\eject
	Prikaz stranice:\\
	
	\includegraphics[width=\columnwidth]{novost}
	\begin{center}
		\textit{Slika 5.22}
	\end{center}
	\eject
	
	Test 5: ispitivanje funkcionalnosti dnevnih taktika\\
	Preduvjeti: Nema\\
	Koraci:\\
	1. Kliknuti na gumb za početak dnevne taktike\\
	2. Pomaknuti figurice po želji\\
	3. Kliknuti na gumb za završetak\\
	
	Očekivani rezultat: Ako je kosrisnik prijavljen te je točno riješio taktiku, bit će stavljen na rang listu \\
	Stvarni izlaz: Taktika je pogrešno riješena\\
	
	Prikaz rezultata za registraciju u Selenium IDE-u:\\
	
	\includegraphics[width=\columnwidth]{dnevnaTaktika}
	\begin{center}
		\textit{Slika 5.23}
	\end{center}
	\eject
	Prikaz stranice:\\
	
	\includegraphics[width=\columnwidth]{taktika}
	\begin{center}
		\textit{Slika 5.24}
	\end{center}
	\eject
	
	
	
	\subsection{Dijagram razmještaja}
	Na slici 5.? prikazan je dijagram razmještaja koji opisuje topologiju sustava. Sustav se sastoji od poslužiteljskog računala na kojem se nalaze web poslužitelj te poslužitelj baze podataka. Na dijagramu je prikazano i klijentsko računalo, a klijent pristupa aplikaciji preko web preglednika. Komunikacija klijent – poslužitelj odvija se preko HTTP protokola.\\
	\\
	\includegraphics[width=\columnwidth]{razmjestaj}
	\begin{center}
		\textit{Slika 5.25: Dijagram razmještaja}
	\end{center}
	\eject
	\subsection{Upute za puštanje u pogon}
	Web aplikaciju koju smo razvili u okviru ovog projekta pustili smo u pogon koristeći Render\footnote{\url{https://render.com/}} web stranicu koja nudi besplatne usluge dovoljne za funkcionalnost cijele naše aplikacije. \\
	Na Render-u smo postavili PostgreSQL bazu podataka, Node poslužitelj koji odgovara na upite korisnika te statičku stranicu koja služi za prikaz React web aplikacije. \\
	\\
	\includegraphics[width=\columnwidth]{1}
	\begin{center}
		\textit{Slika 5.26}
	\end{center}
	\textbf{Postavljanje baze podataka}\\
	Postavljanje baze podataka kroz Render web sučelje je jednostavno. Render nam nudi biranje određenih parametara baze poput imena, korisnika baze, lozinke te vrata baze putem kojih se Node poslužitelj spaja na bazu. Nakon što smo napravili bazu podataka ona je prazna te u nju moramo dodati tablice prilagođene našoj web aplikaciji. Render nam za to daje PSQL naredbu putem koje se možemo izvana spojiti na bazu podataka te ju onda napuniti podatcima. 
	\includegraphics[width=\columnwidth]{2}
	\begin{center}
		\textit{Slika 5.27}
	\end{center}
	
	Iduća slika prikazuje naredbeni redak putem kojeg punimo bazu željenim tablicama.\\
	\includegraphics[width=\columnwidth]{10}
	\begin{center}
		\textit{Slika 5.28}
	\end{center}

	
	Podatci o svim tablicama baze podataka nalaze se u datoteci seed.js. Ta datoteka sadrži SQL naredbe putem kojih direktno inicijaliziramo bazu. Pomoću PSQL naredbe se iz naredbenog retka izvana spajamo na bazu te kopiramo sve naredbe iz seed.js. Nakon ovog koraka u bazi se nalaze sve tablice potrebne za rad web aplikacije. \\
	
	\textbf{Postavljanje Node poslužitelja}\\
	Kroz Render web sučelje stvaramo poslužitelj koji će odgovarati na upite korisnika iz React web aplikacije te koji komunicira sa bazom podataka. Renderova usluga Web service je pravi izbor za ovo. Render nam nudi odabir GitLab grane koja će služiti za deploy poslužitelja te konfiguriranje putanje u kojoj se nalaze datoteke poslužitelja. Odabrali smo main granu našeg projekta te sada svaki puta kada se dogodi neki push na tu granu Node poslužitelj na Renderu će se automatski ažurirati. \\
	
	\includegraphics[width=\columnwidth]{4}
	\begin{center}
		\textit{Slika 5.29}
	\end{center}

	
	Također, moramo postaviti i naredbu putem koje će Render pokretati poslužitelj. Kako je riječ o Node poslužitelju to radimo sa naredbom: node server.js. \\
	
	\includegraphics[width=\columnwidth]{5}
	\begin{center}
		\textit{Slika 5.30}
	\end{center}

	Kako bi sve radilo moramo konfigurirati i varijable okoline koje se ne objavljuju na GitLab već je svaki član tima imao vlastitu .env datoteku koju je konfigurirao prema vlastitim potrebama. U .env datoteci sada se moraju nalaziti parametri koji će omogućiti Node poslužitelju spajanje na prethodno stvorenu bazu i stvaranje Json web tokena.\\
	
	 \includegraphics[width=\columnwidth]{3}
	 \begin{center}
	 	\textit{Slika 5.31}
	 \end{center}
 
 	\textbf{Postavljanje React web aplikacije}\\
 	Stvaranje React web aplikacije kroz Render je jednostavno. React web aplikaciju kroz Render postavljamo tako da koristimo static site uslugu. Kao i kod kofiguriranja Node poslužitelja biramo GitLab granu te putanju do datoteka potrebnih za React web aplikaciju.
 	
 	\includegraphics[width=\columnwidth]{6}
 	\begin{center}
 		\textit{Slika 5.32}
 	\end{center}
 	
 	Kako bi Render znao stvoriti build verziju React web aplikacije moramo napisati ispravnu naredbu. \\
 	
 	 	\includegraphics[width=\columnwidth]{7}
 	\begin{center}
 		\textit{Slika 5.33}
 	\end{center}
 	
 	
	Ovom naredbom React stvara optimalnu verziju web aplikacije koju će Render na kraju i prikazivati u pregledniku. Optimalnu verziju čini jedan html dokument te nekoliko Javascript datoteka potrebnih za rad aplikacije. Datoteke optimalne verzije tj. builda nalaze se u direktoriju build. \\
	
		\includegraphics[width=\columnwidth]{8}
	\begin{center}
		\textit{Slika 5.34}
	\end{center}
	
	
	Ostalo je još konfigurirati varijable okoline. Sada imamo jednu naredbu koja predstavlja URL kojim React web aplikacija pristupa Node poslužitelju. \\
	
	\includegraphics[width=\columnwidth]{9}
	\begin{center}
		\textit{Slika 5.35}
	\end{center}

	
	Nakon svih ovih koraka aplikaciji pristupamo putem URL-a koji nam je dao Render za static site uslugu. Ovim URL-om pristupamo React web aplikaciji koja šalje upite na Node poslužitelj koji dalje može komunicirati sa bazom. 
	\eject
	
	
	
	\section{Zaključak i budući rad}
	Nakon 17 tjedana postignut je konačni cilj ovog projekta a to je izrada funkcionalne web aplikacije šahovskog kluba. U ostvarenoj aplikaciji omogućeno je registriranje i prijava korisnika, rješavanje dnevnih taktika, pregled rang listi, prijavljivanje pogrešaka u dnevnoj taktici te revizija istih, plaćanje članarine šahovskog kluba, prijavljivanje na treninge i turnire postavljenih na aplikaciju od strane ovlaštenih osoba te pregled osobnih podataka.\\
	Sama izrada aplikacije tekla je u dvije faze koje su se otprilike podudarale s ciklusima akademskog semestra.\\
	Prva faza bila je slabijeg intenziteta od druge no nije bila bez svojih izazova. U prvoj fazi bilo je ključno definirati potrebne stvari za izradu aplikaciju te konceptualno izgraditi "kostur" prema kojem će se graditi aplikacija. U tu svrhu trebalo je dobro razumjeti i dokumentirati sve zahtjeve naše aplikacije, izraditi use case-ove te njihove dijagrame, izraditi dijagrame razreda te osmisliti bazu podataka. Sve je to bilo potrebno lijepo dokumentirati u jedan čitki i smisleni dokument čije poglavlje upravo čitate. Osim dokumentacije, potrebno je bilo implementirati generičke funkcionalnosti stranicu kao što su registracija, prijava i slično. A uz sve to, svi smo se morali prilagoditi radu u grupi sa osobama koje smo uglavnom netom upoznali. \\
	Nakon uspješno i kvalitetno odrađene prve faze, uslijedila je druga, užurbanija i zahtjevnija faza. U ovoj fazi glavni fokus bio je na izradi implementacijskih rješenja. Na trenutke je ovo iziskivalo mnoštvo napora s obzirom da je većini članova ovo prvi susret s radom na projektu ovakvog obujma. No ipak sve je uglavnom uspješno savladano te smo na kraju dobili aplikaciju koja je spremna za korištenje, sve što joj je potrebno je stvarni šahovski klub. Osim same implementacije, bilo je i posla oko ispitivanje programskog rješenja te puštanja aplikacije u pogon(deployment). Također, na dokumentaciji je preostalo posla, kao što je izrada dijagrama aktivnosti, stanja, komponenti, razmještaja i slično.\\
	Iako je aplikacija izrađena i puštena u pogon prostora za daljnji rad ima. Sitne greške se mogu pronaći te bi ih se moglo još dodatno "ispeglati". Na dizajnu stranice bi se također dalo poraditi, no i ovakvim minimalističkim stilom i dalje izgleda pristojno. Jedan od mogućih daljnjih razvoja je i omogućavanje igranja partija tzv. dopisnog šaha u kojem članovi međusobno naizmjence šalju poteze jedan drugome unutar nekog definiranog vremenskog okvira te na taj način simuliraju "over the board" partiju šaha. Također bi se moglo poraditi i na dodavanju video sadržaja od strane trener, na primjer lekcije o otvaranjima, analize legendarnih partija i slično.\\
	Ovaj projekt te naš rad i sudjelovanje vrijedno je iskustvo za sve članove tima. Nije bilo lako prilagoditi se na ovakav način rada te obujam posla, no to je ono što nas sve čeka u budućnosti. Timski rad, podrška drugima, organiziranje vremena, samoedukacija i korištenje do sada nepoznatih tehnologija; sve su to vrline koje smo trebali usvojiti kako bi odradili ovaj projekt. Uzimajući u obzir sve navedeno, ali i ostale obveze koje individualno imamo na nekim drugim područjima, možemo biti zadovoljni s obavljenim poslom na ovom projektu.\eject
	\eject
	
	\section{Popis literature}
	\begin{enumerate}
		\item Programsko inženjerstvo, FER ZEMRIS, \url{http://www.fer.hr/predmet/proinz}
		\item The Unified Modelling Language, \url{https://www.uml-diagrams.org/}
		\item Astah Community, \url{http://astah.net/editions/uml-new}
		\item React MUI, \url{https://mui.com/}
	\end{enumerate}
	\eject
	\section{Dodatak: Prikaz aktivnosti grupe}
	\subsection{Dnevnik sastajanja}
	\begin{packed_enum}
		\item sastanak
		
		\item[] \begin{packed_item}
			\item Datum: 17. listopada 2022.
			\item Prisustvovali: Ivan Bilobrk, Altea Božić, Tina Jureško, Lara Mahalec, Mijo Rajič, Ilan Vezmarović i Anteo Vukasović
			\item Teme sastanka:
			\begin{packed_item}
				\item  Upoznavanje tima
				\item  Okvirna podjela poslova
			\end{packed_item}
		\end{packed_item}
		
		\item sastanak
		\item[] \begin{packed_item}
			\item Datum: 24. listopada 2022. 
			\item Prisustvovali:  Ivan Bilobrk, Altea Božić, Tina Jureško, Lara Mahalec, Mijo Rajič, Ilan Vezmarović i Anteo Vukasović
			\item Teme sastanka:
			\begin{packed_item}
				\item  Diskutirana izrada funkcionalnih zahtjeca, use case zahtjeva, sekvencijskih dijagrama te ostalih zahtjeva
				\item  Obavljena konkretna podjela posla
			\end{packed_item}
		\end{packed_item}
	
		\item sastanak
		\item[] \begin{packed_item}
		\item Datum: 4. studenog 2022. 
		\item Prisustvovali:  Ivan Bilobrk, Altea Božić, Tina Jureško, Lara Mahalec i Mijo Rajič
		\item Teme sastanka:
		\begin{packed_item}
			\item  Provjera obavljenog posla
			\item  Razrada baze podataka i početak njene implementacije
			\item  Podjela daljnjeg rada u dokumentaciji
		\end{packed_item}
	\end{packed_item}

		\item sastanak
		\item[] \begin{packed_item}
			\item Datum: 9. studenog 2022. 
			\item Prisustvovali:  Ivan Bilobrk, Altea Božić, Tina Jureško, Lara Mahalec, Mijo Rajič i Anteo Vukasović
			\item Teme sastanka:
			\begin{packed_item}
				\item  Prikaz i testiranje do sada obavljenog posla
			\end{packed_item}
		\end{packed_item}
	
	\eject
	
	\item sastanak
	\item[] \begin{packed_item}
		\item Datum: 18. studenog 2022. 
		\item Prisustvovali:  Ivan Bilobrk, Altea Božić, Tina Jureško, Lara Mahalec, Mijo Rajič, Ilan Vezmarović i Anteo Vukasović
		\item Teme sastanka:
		\begin{packed_item}
			\item  Provjera i potvrđivanje napisane dokumentacije za prvi ciklus
		\end{packed_item}
		\end{packed_item}
	 	\item sastanak
		\item[] \begin{packed_item}
			\item Datum: 9. prosinca 2022. 
			\item Prisustvovali:  Ivan Bilobrk, Altea Božić, Tina Jureško, Lara Mahalec, Mijo Rajič, Ilan Vezmarović i Anteo Vukasović
			\item Teme sastanka:
			\begin{packed_item}
				\item  Podjela poslova implementacije među članovima tima
			\end{packed_item}
		\end{packed_item}
		\item sastanak
		\item[] \begin{packed_item}
			\item Datum: 14. prosinca 2022. 
			\item Prisustvovali:  Ivan Bilobrk, Altea Božić, Tina Jureško, Lara Mahalec, Mijo Rajič, Ilan Vezmarović i Anteo Vukasović
			\item Teme sastanka:
			\begin{packed_item}
				\item  Konzultacije s asistentom
			\end{packed_item}
		\end{packed_item}
		\item sastanak
	\item[] \begin{packed_item}
		\item Datum: 22. prosinca 2022. 
		\item Prisustvovali:  Ivan Bilobrk, Altea Božić, Tina Jureško, Lara Mahalec, Mijo Rajič, Ilan Vezmarović i Anteo Vukasović
		\item Teme sastanka:
		\begin{packed_item}
			\item  Prikaz i testiranje implementiranog sadržaja
			\item Podjela daljnjih poslova među članovima tima 
		\end{packed_item}
	\end{packed_item}
		\item sastanak
	\item[] \begin{packed_item}
		\item Datum: 4. siječnja 2023. 
		\item Prisustvovali:  Ivan Bilobrk, Altea Božić, Tina Jureško, Lara Mahalec, Mijo Rajič i Anteo Vukasović
		\item Teme sastanka:
		\begin{packed_item}
			\item  Prikaz i testiranje do sada obavljenog posla
			\item  Planiranje rada završne faze implementacija
			\item  Podjela daljnjeg rada u dokumentaciji
			\item  Podjela posla za ispitivanje programskog rješenja
		\end{packed_item}
	\end{packed_item}
	\eject
		\item sastanak
	\item[] \begin{packed_item}
		\item Datum: 13. siječnja 2022. 
		\item Prisustvovali:  Ivan Bilobrk, Altea Božić, Tina Jureško, Lara Mahalec, Mijo Rajič, Ilan Vezmarović i Anteo Vukasović
		\item Teme sastanka:
		\begin{packed_item}
			\item  Konačna verifikacija dokumentacije i programskog rješenja
			\item  Podjela posla oko prezentacije projekta
		\end{packed_item}
	\end{packed_item}
	\end{packed_enum}
	\eject
	\subsection{Tablica aktivnosti}
	U nastavku je prikazana tablica aktivnosti na projektu. Sve vrijednosti tablice predstavljaju sate rada člana tima na pojedinoj aktivnosti.
	\\
	\begin{longtblr}[
		label=none,
		]{
			vlines,hlines,
			width = \textwidth,
			colspec={X[7, l]X[1, c]X[1, c]X[1, c]X[1, c]X[1, c]X[1, c]X[1, c]}, 
			vline{1} = {1}{text=\clap{}},
			hline{1} = {1}{text=\clap{}},
			rowhead = 1,
		} 
		\multicolumn{1}{c|}{} & \multicolumn{1}{c|}{\rotatebox{90}{\textbf{Ivan Bilobrk}}} & \multicolumn{1}{c|}{\rotatebox{90}{\textbf{Altea Božić }}} &	\multicolumn{1}{c|}{\rotatebox{90}{\textbf{Tina Jureško }}} & \multicolumn{1}{c|}{\rotatebox{90}{\textbf{Lara Mahalec }}} &	\multicolumn{1}{c|}{\rotatebox{90}{\textbf{Mijo Rajič }}} & \multicolumn{1}{c|}{\rotatebox{90}{\textbf{Ilan Vezmarović }}} &	\multicolumn{1}{c|}{\rotatebox{90}{\textbf{Anteo Vukasović }}} \\  
		Upravljanje projektom 		& 8 &  &  &  &  &  & \\ 
		Opis projektnog zadatka 	&  &  &  &  &  &  & 1\\ 
		
		Funkcionalni zahtjevi       & 1 & 1 & 1 & 1 & 1 & 1 & 3 \\ 
		Opis pojedinih obrazaca 	& 1 & 1 & 1 & 1 & 1 & 1 & 3 \\ 
		Dijagram obrazaca 			& 1 & 1 & 1 & 1 & 3 & 1 & 1 \\ 
		Sekvencijski dijagrami 		&  & 3 & 3 & 2 &  &  &  \\ 
		Opis ostalih zahtjeva 		&  &  &  &  &  &  & 1 \\ 
		
		Arhitektura i dizajn sustava	 & 1 & 1 & 1 & 1 & 1 & 1 & 2 \\ 
		Baza podataka				& 1 & 1 & 1 & 10 & 1 & 1 &   \\ 
		Dijagram razreda 			&  & 3 & 2 &  & 3 &  & 1  \\ 
		Dijagram stanja				&  & 4 &  &  &  &  &  \\ 
		Dijagram aktivnosti 		&  &  & 4 &  &  &  &  \\ 
		Dijagram komponenti			&  & 4 &  &  &  &  &  \\ 
		Korištene tehnologije i alati 		&  &  &  &  &  &  & 1 \\ 
		Ispitivanje programskog rješenja 	&  &  &  & 8 &  &  &  \\ 
		Dijagram razmještaja			&  &  & 4 &  &  &  &  \\ 
		Upute za puštanje u pogon 		& 1 &  &  &  &  &  &  \\   
		Zaključak i budući rad 		&  &  &  &  &  &  & 1 \\  
		Popis literature 			&  &  &  &  &  &  &  \\  
		Front-end  			& 14 & 10 & 10 & 10 & 9 & 14 & 16 \\  
		Back-end 			& 24 &  &  & 16 & 12 & 18 & 2 \\   
		Deployment 			& 4 &  &  &  &  &  &  \\
	\end{longtblr}
		
	
\end{document}